\section{Pulse Height Analysis}

The amplitude of a pulse generated by a PhotoMultiplier Tube
is directly proportional to the energy carried by a Runaway 
Electron. By measuring the height of the signal peak 
it is thus possible to estimate the RE energy and
produce an X-Ray radiation spectrum. The previous chapter
described methods that can be used for the detection of such pulses.
This section focuses on methods that can be used on top of
the detection algorithms to obtain the best possible resolution 
of the pulse height measurement.

\subsection{Pulse shaping algorithms}
As described in \autoref{ssec:boxcar_filter} the zero crossing point 
of a pulse derivative pinpoints its peak.
The simplest method of obtaining the pulse height
is thus using a derivative and sampling the original signal
at the derivative's zero crossing.
The method works, however it is highly susceptible to noise.
Taking only a single sample grants a result with a precision 
equal to the noise magnitude. It would be beneficial
to take multiple samples of the signal height, to minimize 
the error.


Pulse shaping algorithms help reduce the influence of noise
on the peak height measurement. Analogue pulse processing 
systems typically relied on CR-RC filters to shape 
pulses to a more Gaussian shape. This granted them 
a smoother peak that could be averaged over multiple samples.
In the digital domain a trapezoidal filter 
(see \autoref{ssec:trapezoidal_filter})
is the most commonly chosen shaping filter.


When used for pulse detection, the flattop region of
a trapezoid-shaped pulse is of little interest, and 
can be truncated to form a triangular filter. In Pulse Height
Analysis, the near constant signal produced
by the trapezoid flattop is a crucial feature. 
Instead of taking a single sample at the sharp peak,
multiple samples are taken throughout the whole flattop region.
Although a longer flattop region provides better noise immunity, 
it cannot be increased too much due to pile-ups.

\subsection{Integration}

The pulses generated by a PMT-preamplifier combination
have a constant decay time and differ in amplitude.
They can be fairly accurately modelled with a 
single exponential formula: $A {\rm e}^{-t}$, where $A$ is
the pulse amplitude, and $t$ expresses time, assuming that 
the pulse peak occurs at $t=0$. The surface area
under the curve given by that formula is proportional 
to the amplitude $A$.


The pulses can be integrated to obtain a value that is proportional
to the pulse amplitude and RE energy. As multiple samples
over the pulse duration are accumulated, the effect of noise
is minimized, just like with trapezoidal shaping.
Integration is simpler to understand and implement
than shaping and timing filters. It does not require
the precise decay time to be known. This is beneficial 
as a in the system used in this work the 
single-exponential function is only an approximation 
of the preamplified signal.


While integration only requires the use of a single accumulator 
to add the samples up, it results in a numerically large final sum. 
To obtain a usable result this value must be scaled down to a usable range.
Optimally, a range corresponding to the ADC codes is chosen.
An integral of a pulse is then mapped to value equal to its amplitude.
The constant divider required for the mapping
can be calculated for a fixed sampling window length
or obtained experimentally. 


The biggest drawback of using integration is its 
susceptibility to pile-up effects. To obtain accurate results
in environments with high pile-up or long pulse decay, complicated 
pile-up compensation must be used. These algorithms 
work to subtract or otherwise remove already detected pulses
from the input signal.
Although such solutions have been proposed in research work,
their implementation is expensive and usually limited to a single
pile-up of just two pulses. \cite{pileup_correction}

\subsection{Simulated performance}
