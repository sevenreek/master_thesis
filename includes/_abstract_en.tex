In the last few decades ultra-high speed digitizers and 
Field Programmable Gate Arrays have made it possible to shift
to Digital Signal Processing in nearly all experiments and applications
requiring nanosecond precision. 
The drawback of using digital processing at sampling rates in the gigahertz range
is that it produces a very high data throughput. Gigabytes of data
are generated by such systems every second. Real-time analysis 
of these signals is a challenge that requires careful optimization
at multiple levels of the processing pipelines.


In fusion reactors one of the diagnostic systems that requires processing
at such precision is the Hard X-Ray Monitor. 
This work describes a part of a Hard X-Ray Monitor system developed for 
the ITER tokamak project, with a focus on real-time firmware processing
of PhotoMultiplier Tube signals in a Field Programmable Gate Array.
Different algorithms for pulse detection are described, compared 
in theory and simulated in software.
The results are analyzed to find a compromise between complexity and accuracy.
Best methods are implemented in an FPGA and tested with a real radioactive sample.
Same approach is taken with digital shaping and Pulse Height Analysis.
The combined algorithms form a real-time spectroscopy solution.


The Hard X-Ray Monitor developed in this work can successfully 
detect and categorize pulses produced by a PMT, even when 
run at the maximum digitizer speed of 1 GHz. No dead-time is 
introduced thanks to the use of custom binning and storage algorithms.
The firmware implementation of the solution used for the generation and storage of
the X-Ray spectra is described. The current system shortcoming in regards to ITER's 
requirements are outlined together with a plan for future improvements.
Additionally, requirements for next generation digitizers are given.
Core issues of real-time acquisition are outlined and potential solutions
are proposed and tested in custom benchmarks. Some limitations in
current state-of-the-art digital signal processing are provided.

