\section{HXRM firmware}

Based on the results of research and simulations, the best timing
and shaping algorithms were implemented in hardware.
The ADQ14 uses Verilog and Vivado 2015.2, so these tools
were chosen for the implementation of the HXRM.
The developed system performs pulse detection,
measures the peak height, stores the resultant spectrum
and periodically transfer the results to the host PC.

\subsection{System overview}

As described in \autoref{ssec:adq_devkit} the ADQ14 DevKit
grants a semi-open FPGA design with two modules that can be
freely modified by the end user. The board manufacturer
designed User Logic 1 with intent for it to house timing filters
and User Logic 2 to contain more specialized processing logic.
Initially, these ideas were followed with a Boxcar filter being
implemented in User Logic 1 and a Pulse Height Analyzer being placed 
in User Logic 2.


With the default firmware, ADQ14 generates fixed-length records
on trigger events. For example, with the window length set to a 1000
samples, upon the detection of a pulse, 1000 samples would always be
transferred to the PC, irregardless of pile ups and other disturbances.
For this reason the timing logic originally placed in User Logic 1 was moved
to User Logic 2, so that it could be better integrated with the rest 
of the pulse analysis systems. The UL2 module allows for full control 
over record lengths so sampling windows in which pile ups were detected
could now be rejected, split into two or prolonged depending on the need.
\autoref{fig:firmware_overview} gives an overview of the system designed
in User Logic 2.

\subsection{System control}

The system is configured and controlled through so called 
user registers. $2^19$ individually addressable 32-bit words
are available for reading and writing to from the PC. The first
four words are reserved for internal use. Next four have been 
dedicated for configuration and control as listed in \autoref{tab:register_control}.
The $2^14$ block of addresses starting from position 9, can
be used to read values from bins forming the currently held spectrum.

\subsection{Pulse detection}
\subsection{Pulse height analysis}
\subsection{Spectrum storage and transfer}
