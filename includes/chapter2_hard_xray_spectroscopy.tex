\section{Hard X-Ray spectroscopy}

One of the diagnostic systems in a lot of currently existing tokamaks
is the Hard X-Ray Monitor (HXRM). It will also be implemented in ITER.
The device is tasked with measuring the spectrum of X-Ray radiation
inside the fusion vessel. The presence of high energy X-Ray
radiation can point to problems with the plasma stability and suggest the 
need for mitigation techniques.
\cite{low_noise_amplifier_for_pmt}

\subsection{Runaway electrons}

The plasma in a tokamak is heated to a level where particles reach a velocity
enabling them to break through atomic repulsion. In such conditions some 
particles can also obtain sufficient speed to escape the magnetic confinement
of the vessel. Typically collisions with other particles are so frequent
that in stable plasma these disturbances are not happening in large quantities.
Directly after plasma is disrupted or terminated, 
the probability of collisions is lessened 
and a tokamak might start acting like a particle accelerator,
bringing the electrons to nearly the speed of light.
Electrons that act in this manner are called Runaway Electrons (REs).
\cite{iter_re_melt}


This phenomenon can occur in the form of a high-energy 
concentrated beam capable of melting the front-facing walls of a reactor.
The effects of such damage being purposefully introduced in the JET tokamak
are shown in \autoref{fig:re_melt}. To prevent mitigate the destructive
effects of REs their generation has to be avoided if possible. Otherwise
they must be detected and dealt with accordingly. One method of doing so 
involves the injection of noble gases in a process called 
Massive Gas Injection (MGI) \cite{massive_gas_injection}.
\begin{figure}[h]
  \centering
  \includegraphics[width=.7\linewidth]{media/re_melt.jpeg}
  \caption{Effects of RE in the JET vessel\cite{iter_re_melt}}
  \label{fig:re_melt}
\end{figure}


When REs interact with the Plasma Facing Components (PFCs)
they lose their energy and emit X-Ray radiation in 
the Bremsstrahlung process. The energy of this radiation
varies greatly, ranging from tens of keV (Soft X-Ray)
to multiple MeV (Hard X-Ray).
\cite{hxrm_jet}

\subsection{PhotoMultiplier Tubes}

The photons generated in Bremsstrahlung by Runaway Electrons
can be detected with the use of a device known as a PhotoMultiplier Tube (PMT).
A PMT is built with the use of a photocathode and an electron multiplier.
When a radiation photon hits the photocathode an electron is emitted
due to the photoelectric effect. Electric fields in the PMT accelerate the 
electron towards a series of dynodes. Each collision with a dynode causes
a release of additional electrons and forms a stronger beam that 
eventually reaches the anode where it becomes measurable as a current pulse.
Typically a gain in the order of $10^6$ is obtained.
\cite{pmts_basics, pmt_gain}

\subsection{Preamplifiers}

The internal gain of a PMT is sufficient for many applications and 
does not require additional amplification. The produced current impulses 
can be detected with precise circuitry,
however there are multiple reasons for which
a preamplifier is often used in tangent with a PMT. With a typical
load of $50 \Omega$ the output signal of a single photon is
a very sharp voltage peak of around 10 mV.
Unprocessed short PMT pulses are sufficient for detection, counting 
and timing, but may be problematic when it comes to discrimination and 
processes like Pulse Height Analysis (PHA).
\cite{why_pmt_need_amplifiers}


In those scenarios that do not require the sharpest response,
the slight increase in Signal to Noise Ratio is worth the 
features introduced by an amplifier. These can include
impedance matching, filtering and pulse-shaping.
In tokamak applications it also helps move most of the 
diagnostic infrastructure away from the dangerous environment
created by the fusion plasma. 


In ITER the X-Ray radiation will first be converted to a light pulse
using a scintillator. This light will be transferred over a 12 m 
fiber before reaching the PMT. Then the electrical signal from 
the PMT must pass through a 100 m copper coaxial cable, before
reaching the acquisition and processing hardware. The internal 
input of a PMT is thus insufficient in this scenario and must be 
preamplified.
\cite{low_noise_amplifier_for_pmt}

\subsection{Pulse processing chains}

To obtain a radiation spectrum from the voltage pulses,
their height must be measured and
placed into an appropriate bin consisting of a range of voltage levels.
The pulses generated by a PMT last just a few nanoseconds,
making the task at hand complicated.


With a preamplifier this duration is increased by a factor of around
a few hundreds depending on the preamplifier components.
This results in pulses that last a few hundreds nanoseconds.
Before the advent of ultra-high speed Analogue to Digital Converters (ADCs),
such short events could not viably be processed with digital 
electronics and had to rely on analogue components.

\subsubsection{Analogue processing chains}

In analogue radiation spectroscopy
pulses are typically first transformed to a Gaussian shape,
with a series of low- and high-pass filters.
These signals must then pass through complicated pile-up
rejection circuitry. After that the pulses 
are fed to a Multi Channel Analyzer (MCA).
This is the device that performs the binning action.
Initially an MCA would consist of an array of 
analogue comparators, and over time it would rely 
on more and more digital components.
The schematics of a typical analogue system are shown in \autoref{fig:analogue_pp}
\begin{figure}[h]
  \centering
  \includegraphics[width=\linewidth]{media/analog_pulse_processing.png}
  \caption{Analogue Pulse Processing chain}
  \label{fig:analogue_pp}
\end{figure}



As mentioned earlier, for a long time analogue processing 
was the only way to reliably handle events shorter than 100 ns. 
It was, however, quickly recognized
that the digital approach offered 
much lesser susceptibility to outside noise. 
Digital components could also potentially be tuned without
having to physically modify the circuit.
These two features are particularly important in the 
complicated environment of a fusion reactor.
\cite{analog_vs_digital_1998}

\subsubsection{Digital processing chains}

As soon as ADCs and digital processing circuits capable of
reaching the resolution required for precise nuclear
spectroscopy appeared on the market they were 
adopted into new experimental designs of MCAs \cite{mca_fpga}.
As the technology improved ADCs were moved closer to the 
radiation detector itself.
A single reprogrammable silicon chip, together with a fast ADC,
could perform the job of a number of the analogue components
in a spectroscopy system.
On top of that its operation is less susceptible to Electro Magnetic
Interference (EMI) and temperature-induced parameter variance.
\cite{dpp_walewski}


The earlier in a processing chain that the ADC is placed the
lesser the influence of imperfect analogue components is.
Seeing an ADC right after the preamplifier is commonplace
in modern systems as shown in \autoref{fig:digital_pp}.
Such approach, however, produces an important issue.
To obtain a sufficient horizontal resolution comparable to analogue chains
the ADC must sample the signal with a frequency of at least
a few hundred MHz. This means that, at a typical vertical resolution
of anywhere between 8 to 16 bits, a modern high-speed ADC
can generate anywhere between 100 megabytes 
up to even a few gigabytes of data every second.
\cite{dpp_walewski}
\begin{figure}[h]
  \centering
  \includegraphics[width=.7\linewidth]{media/digital_pulse_processing.png}
  \caption{Digital Pulse Processing chain}
  \label{fig:digital_pp}
\end{figure}

Processing gigabytes of data in real time poses one of 
the primary challenges in designing Digital Pulse Processing systems.
Despite that, fast digitizers are used for Hard X-Ray Spectroscopy
in existing tokamaks, like KSTAR or JET \cite{hxrm_jet, kstar_upgrade}.
ITER will require similar or better systems during its operation,
so the problems of handling large data throughput
should be considered solved before the diagnostic systems
are installed in the facility. Once first plasma is obtained,
the possibility of system modifications will be greatly limited.
 


