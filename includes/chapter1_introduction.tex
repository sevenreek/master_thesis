\section{Introduction}


\subsection{Motivation}


Concerns regarding the sustainability of using fossil fuels for energy generation
have been raised as early as the 1970s \cite{rethinking_resource_depletion}. 
One of the most well-known examples from that time was the 1972 report
titled "Limits to Growth" by Meadows et. al. \cite{limits_to_growth}.
In it a group of MIT scientists attempted to answer the question of
how long will the Earth's natural resources last for
considering the seemingly neverending growth of human civilisation.
As a result of a conducted computer simulation,
a rough estimate of around 100 years was given as a timeframe,
after which the population would start to collapse due to a lack of resources.


This estimate did not go without controversies back when it was first published.
The methodology was thoroughly picked apart leading many to dismiss the study findings
\cite{rethinking_resource_depletion}. Naturally, nowadays, we are much better poised to verify
the claims made by the now 50 year old book. The impeding resource depletion
has certainly been made a less valid claim as technological progress
made it possible to locate and tap into previously inaccesible fossil fuel fields
\cite{shaping_the_global_oil_peak}.
Taking into account other issues, however, the original timeline of 100 years might have
actually shifted closer. 


When it comes to fossil fuel usage, in the last twenty years, 
the primary concerns have changed from resource depletion to global warming 
and irreversible environmental damage \cite{rethinking_resource_depletion}.
In 2018 the Intergovermental Panel on Climate Change (IPCC) published a
report indicating the need to stop the global temperature increase 
at 1.5\degree C above the levels measurable in the pre-industrial era.
Failure to do so is projected to lead to irreversible climate changes and in turn
serious damage to human settlements around the world \cite{ipcc2018}.


Fossil fuels account for as much as 70\% of greenhouse gas emissions.
Electricity generation alone causes 25-35\% 
\cite{global_climate_change} of the total amount.
Such a high share means that reducing this output
is going to be crucial in meeting the goals outlined by the IPCC.

\subsection{Fusion energy}


\subsection{Fission energy}


\subsection{ITER tokamak}


\subsection{Field Programmable Gate Arrays}


\subsection{Problem statement}

