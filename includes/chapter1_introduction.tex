\section{Introduction}


\subsection{Motivation}


Concerns regarding the sustainability of using fossil fuels for energy generation
have been raised as early as the 1970s \cite{rethinking_resource_depletion}. 
One of the most well-known examples from that time was the 1972 report
titled "Limits to Growth" by Meadows et. al. \cite{limits_to_growth}.
In it a group of MIT scientists attempted to answer the question of
how long will the Earth's natural resources last for
considering the seemingly neverending growth of human civilisation.
As a result of a conducted computer simulation,
a rough estimate of around 100 years was given as a timeframe,
after which the population would start to collapse due to a lack of resources.


This estimate did not go without controversies back when it was first published.
The methodology was thoroughly picked apart leading many to dismiss the study findings
\cite{rethinking_resource_depletion}. Naturally, nowadays, we are much better poised to verify
the claims made by the now 50 year old book. The impeding resource depletion
has certainly been made a less valid claim as technological progress
made it possible to locate and tap into previously inaccesible fossil fuel fields
\cite{shaping_the_global_oil_peak}.
Taking into account other issues, however, the original timeline of 100 years might have
actually shifted closer. 


When it comes to fossil fuel usage, in the last twenty years, 
the primary concerns have changed from resource depletion to global warming 
and irreversible environmental damage \cite{rethinking_resource_depletion}.
In 2018 the Intergovermental Panel on Climate Change (IPCC) published a
report indicating the need to stop the global temperature increase 
at 1.5\degree C above the levels measurable in the pre-industrial era.
Failure to do so is projected to lead to irreversible climate changes and in turn
serious damage to human settlements around the world \cite{ipcc2018}.


Fossil fuels account for as much as 70\% of greenhouse gas emissions.
Electricity generation alone causes 25-35\% 
\cite{global_climate_change} of the total amount.
Such a high share means that reducing this output
is going to be crucial in meeting the goals outlined by the IPCC.
At the beginning of the twentieth century, renewable energies, i.e. 
wind, solar, biomass and geothermal were thought
to be the perfect solution to the issue at hand
\cite{renewable_review_2000}. 


In modern times, we have now become aware of multiple issues
that make renewable energy generation a problem at large scale.
Most importantly, their efficacy varies depending on the geographical
location and climate. Even when placed in optimal conditions,
they do not offer perfect stability. Additionally, the land
usage is greater than the traditional forms of energy production
\cite{renewable_problems}.

\subsection{Fission energy}

The drawbacks of renewable energies have led to a formation
of an alternative approach in both research and policymaking. 
The use of nuclear energy for supplementing the shortcomings 
of renewables has been suggested as a potential path forward.
This concept is referred to as hybrid nuclear-renewable system.
\cite{hybrid_nuclear_renewable}. 

There are two ways that nuclear energy can be created and harnessed.
In the more well-established technology, fission, heavy atoms 
(usually Uranium) are bombarded with neutrons 
and split into two or more lighter nuclei and
additional neutrons. The reaction is self-sustaining 
and releases energy in the form of heat that is then used
to boil water. The steam causes turbines to spin
and generate electricity.


Fission is far from a new concept, as first fission reactors have been 
built as early as 1942 \cite{first_fission_reactor}. 
Although the technology itself is quite old 
and has been greatly improved over time, 
there is reasonable reluctance to build and use fission power plants. 
The issue that gets raised most often is the storage of radioactive waste. 
There are, however, multiple less well-known
problems with fission \cite{fission_problems}. 


The tragedies of Chornobyl and Fukushima reactors
have caused many people to be wary of fission. However, even if
democratic support is disregarded in policymaking, the acquisition,
storage and disposal of radioactive materials required for and produced 
during fission prove to be an administrative challenge, especially 
if reactor construction and maintenance is to be handled
by private entities \cite{fission_tech_and_current_issues}. 
The complexity of the problem suggests that 
as we arrive to more concrete solutions we 
should not stop exploring other potential alternatives.


\subsection{Fusion energy}

Just like it is possible to split atoms, it is also possible to
combine them together in a process referred to as fusion. 
What is more, by fusing atoms lighter than Iron
the reaction can also produce surplus energy,
that can be used to generate electricity. 
The conditions necessary for fusion to happen are 
extremely harder to achieve and then sustain 
\cite{structural_materials_fusion}.


Fusion is the primary reaction that cause stars to emit energy.
The fusion that is artificially attempted on Earth differs in the input 
components from that occurring naturally in the Sun. 
A p-p reaction, where 4 protons are converted into ${}^{4}$He.
Replicating this reaction on a larger scale is extremely challenging 
due to the need to convert protons into neutrons.
Our fusion experiments primarily rely on using hydrogen isotopes, 
most commonly deuterium (D) and tritium (T).


Despite being an easier approach, it still requires us to sustain
a 200 million \degree C plasma. This means that an enormous amount of energy
must be used to first heat the plasma up and then confine it to 
prevent it from completely destroying the reactor. 
The efficiency of D-T reactions might, however, worth the trouble.
Theoretically, just 30 mg of deuterium would generate as much energy
as 250 l of gasoline \cite{nuclear_fusion_status}. 


Such numbers sound incredible, but there are naturally multiple drawbacks too.
Tritium, the other input material of this most promising reaction is
extremely rare in nature. Its artificial production is currently 
done only by a select number of facilities. 
Combined with its relatively short half-life of around 12 years, 
there are fears of it running out. It is proven that fusion reactors
will be capable of "breeding" their own tritium, however the transition period 
may still prove to be troublesome \cite{fusion_fuel_running_out}.


In the end, despite being a similarly old technology as fission \cite{fusion_history},
a fusion reactor with a net positive energy balance
has not yet been constructed. Containing plasma heated to such extreme
temperatures cannot be achieved with any solid material and must 
be done with the use of inertial or magnetic forces. 
The most common reactors that employ this concept are:
tokamaks and stellarators. The former design
has been selected for probably the most ambitious fusion project to date, 
the International Thermonuclear Experimental Reactor \cite{nuclear_fusion_status}.


\subsection{ITER tokamak project}


\subsection{Field Programmable Gate Arrays}


\subsection{Problem statement}

