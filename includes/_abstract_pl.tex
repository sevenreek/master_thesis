W ciągu ostatnich dziesięcioleci większość analogowych komponentów
w systemach do akwizycji sygnałów została zastąpiona przez 
elementy cyfrowe. Cyfrowe przetwarzanie sygnałów oferuje
wiele zalet w stosunku do analogowych metod.
Przetworniki cyfrowe są mniej podatne na temperaturę i 
zakłócenia elektromagnetyczne.
Niestety, przed pojawieniem się dostatecznie szybkich przetworników analogowo-cyfrowych
na rynku konsumenckim, cyfrowe przetwarzanie sygnałów nie mogło być 
efektywnie wykorzystywane w eksperymentach wymagających precyzji czasowe
lepszej niż nanosekunda. 
Obecnie urządzenia tego typu są dostępne, jednak ich głównym problemem  
jest bardzo wysoka przepustowość danych. Najszybsze przetworniki
są w stanie generować gigabajty danych na sekundę.


W eksperymentach z dziedziny fuzji jądrowej powstają sygnały
o długości od kilkudziesięciu do kilkuset nanosekund, w związku
z czym konieczne jest zastosowanie w nich szybkich przetworników.
Łatwość rekonfiguracji urządzeń cyfrowych sprawia,
że są one preferowanym rozwiązaniem w wysoce eksperymentalnych
zastosowaniach. Aby móc polegać na cyfrowym przetwarzaniu
w fuzji, problemy wynikające z wysokiej przepustowości 
urządzeń cyfrowych muszą zostać rozwiązane, zanim aktualnie konstruowane
reaktory wejdą w fazę testów z plazmą.


Jednym z systemów diagnostycznych powszechnie wykorszystywanym w 
tokamakach jest Monitor Twardego Promieniowania X (Hard X-Ray Monitor).
System ten, mierzy spektrum promieniowania produkowane przez tzw. Runaway
Electrons. Wczesne wykrywanie i redukcja tego zjawiska są kluczowe
dla bezpieczengo działania tokamaka.
Monitory Promieniowania X wykorzystują fotopowielacze (PhotoMultiplier Tubes)
do mierzenia wiązek promieniowania X, powstających w wyniku hamowania
elektronów na wewnętrznych ścianach reaktora.


W tej pracy opisana została część Monitora Twardego Promieniowania X 
zaprojektowanego dla projektu ITER, z uwagą skupioną, na cyfrowym przetwarzaniu
impulsów z fotopowielaczy, w czasie rzeczywistym, z wykorzystaniem bezpośrednio
programowalnych macierzy bramek (FPGA). Różne algorytmy do wykrywania 
impulsów zostały opisane, porównane i przetestowane w symulacjach komputerowych.
Algorytmy gwarantujące najlepszy kompromis między łatwością implementacji 
a celnością zostały zaimplementowane w macierzy bramek i przetestowane 
z radioaktywną próbką Cezu. Taki sam proces zastosowano w przypadku
algorytmów do pomiaru wysokości impulsów. Połączone algorytmy wykorzystano
do spektroskopii czasu rzeczywistego.


Zaprojektowany w tej pracy system diagnostyczny jest w stanie kategoryzować 
impulsy z fotopowielaczy, nawet przy maksymalnej prędkości próbkowania 
wykorzystanej płytki do akwizycji, 1 GHz. W procesie,
dzięki wykorzystanym algorytmom przechowywania i
przesyłania spektrum promieniowania, nie występuje czas martwy.
Dodatkowo, przedstawione zostały wymagania względem płytek do akwizycji oraz wyzwania
związane z utrzymywaniem wysokiej przepustowości danych w różnych elementach systemu
do akwizycji danych cyfrowych.

