W ciągu ostatnich dziesięcioleci ultraszybkie przetworniki analogowo-cyfrowe
oraz systemy FPGA umożliwiły wykorzystanie cyfrowego przetwarzania sygnałów 
w praktycznie każdym zastosowaniu wymagającym nawet nanosekundowej precyzji.
Głównym problemem cyfrowego przetwarzania przy częstotliwości próbkowania w 
zakresie gigahertzów jest bardzo wysoka przepustowość danych. Najszybsze przetworniki
są w stanie generować gigabajty danych na sekundę. Analiza sygnałów tego typu 
w czasie rzeczywistym wymaga szczegółowej optymalizacji wszystkich części 
systemu do akwizycji.


W eksperymentach z dziedziny fuzji termojądrowej jednym z systemów diagnostycznych
wymagających przetwarzania z tak wysoką precyzją jest Monitor Twardego Promieniowania X.
W tej pracy opisana została część Monitora Twardego Promieniowania X 
zaprojektowanego dla projektu ITER, z uwagą skupioną, na cyfrowym przetwarzaniu
impulsów z fotopowielaczy, w czasie rzeczywistym, z wykorzystaniem bezpośrednio
programowalnych macierzy bramek (FPGA). Różne algorytmy do wykrywania 
impulsów zostały opisane, porównane i przetestowane w symulacjach komputerowych.
Algorytmy gwarantujące najlepszy kompromis między łatwością implementacji 
a skutecznością detekcji zostały zaimplementowane w macierzy bramek i przetestowane 
z radioaktywną próbką Cezu. Taki sam proces zastosowano w przypadku
algorytmów do pomiaru wysokości impulsów. Połączone algorytmy wykorzystano
do opracowania systemu do spektroskopii czasu rzeczywistego.


Zaprojektowany w tej pracy system diagnostyczny jest w stanie kategoryzować 
impulsy z fotopowielaczy, nawet przy maksymalnej częstotliwości próbkowania 
oferowanej przez wykorzystaną płytkę do akwizycji, 1 GHz. W zaprojektowanym 
procesie spektroskopii, dzięki wykorzystanym algorytmom przechowywania i
przesyłania spektrum promieniowania, nie występuje czas martwy.
W pracy szczegółowo opisano algorytm pozwalający na osiągniecie tych cech.
Dodatkowo, przedstawione są wymagania względem płytek do akwizycji oraz wyzwania
związane z utrzymywaniem wysokiej przepustowości danych w różnych elementach systemu
do akwizycji danych cyfrowych. 

